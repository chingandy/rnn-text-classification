
% last updated in April 2002 by Antje Endemann
% Based on CVPR 07 and LNCS, with modifications by DAF, AZ and elle, 2008 and AA, 2010, and CC, 2011; TT, 2014; AAS, 2016

\documentclass[runningheads]{llncs}
\usepackage{graphicx}
\usepackage{amsmath,amssymb} % define this before the line numbering.
\usepackage{ruler}
\usepackage{color}
\usepackage[width=122mm,left=12mm,paperwidth=146mm,height=193mm,top=12mm,paperheight=217mm]{geometry}

\begin{document}
% \renewcommand\thelinenumber{\color[rgb]{0.2,0.5,0.8}\normalfont\sffamily\scriptsize\arabic{linenumber}\color[rgb]{0,0,0}}
% \renewcommand\makeLineNumber {\hss\thelinenumber\ \hspace{6mm} \rlap{\hskip\textwidth\ \hspace{6.5mm}\thelinenumber}}
% \linenumbers
\pagestyle{headings}
\mainmatter
\def\ECCV16SubNumber{***}
\title{DD2424 Project - character-level text classification with different RNN architectures}

\maketitle

%The final report should include the following sections:

\begin{abstract}
% • Abstract: Where you give an overview of the task and the findings
% of your work in a nutshell.

\dots
\keywords{We would like to encourage you to list your keywords within
the abstract section}
\end{abstract}


\section{Introduction}

% • Introduction/Problem formulation: Motivate the problem you
% are trying to solve, attempt to make an intuitive description of the
% problem and also formally define the problem. (1-2 pages including
% title, authors and abstract)

\section{Background}

% • Background: summarize a few notable approaches/papers tackling
% the same problem. The selection should cover different possible tech-
% niques that can be (have been) used for the same task with success.
% Also, it is good to mention other recognition/synthesis tasks that use
% the same deep learning technique as yours. (1-2 pages)


\section{Approach}
% • Approach: Describe the final approach you are take for this problem.
% For instance, here you would describe the details of the network’s
% architecture. What training parameters and techniques you have used.
% The computational complexity of your model. And similar questions.
% To help explain your approach please make figures to accompany your
% text description. (1-3 pages)

\section{Experiments}
\section{Results}
\section{Conclusions}

% • Experiments/Results/Conclusions: In this section, you should
% present the results you achieved with various experiments. The re-
% sults can be presented in tables, plots, etc. Explain what conclusions
% you can draw from these set of experiments? The set of experiments
% and results reported here should justify some of the design choices
% described in the previous sections. (3-6 pages)

As \cite{Alpher04} said.

% • References: It is extremely important to make sure all the content
% from other sources and the ideas that you build on are properly cited.


% Both positive and negative results should be reported. A discussion re-
% garding why certain techniques worked better than the others is necessary.
% Students are also encouraged to take initiatives in trying out new techniques,
% beyond those discussed at the lectures.
% The stated number of pages above is a guideline, one can go beyond that or
% slightly below. The whole report should be between 7-14 pages.


\bibliographystyle{splncs}
\bibliography{egbib}
\end{document}
